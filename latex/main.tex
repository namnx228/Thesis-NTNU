\documentclass[USenglish]{ttm4502}
% Available languages: USenglish (default), UKenglish, norsk, nynorsk
\usepackage{hyperref}
\begin{document}

\title{Comparison between different container engineers}
\student{Nam Xuan Nguyen} % if more than one use \and; student1 \and student2 \and student3
\supervisor{Danilo Gligoroski} %if more than one use \and; supervisor1 \and supervisor2
\professor{Danilo Gligoroski}
\maketitle

\begin{abstract}
% % A short summary of the project (max 300 words)
% Use for instance http://app.uio.no/ifi/texcount/online.php to count the words
I collect a list of container engineers and aspects to compare in this \textbf{\href{https://htmlpreview.github.io/?https://raw.githubusercontent.com/namnx228/Thesis-NTNU/master/table.html}{link}}.\\
PS: I have not finished the table yet.
\end{abstract}

\section{My project}
\label{sec:my_project}
% Present of the topic area
% Motivate for your project in that area
% formulate objectives, for instance as research questions
% discuss limitations/assumptions/hypotheses, if relevant

\section{Background and related work}
\label{sec:background_and_related_work}
% Present the problem area and the key solutions that have been proposed to solve the same or similar problems
% What are the new elements in what you aim at doing
% It may be relevant to present some of the solutions in details in your master’s thesis, but not here. This does not mean that you should only read few references, but that you should identify keywords, select and sort references and present them in the report in form of a list for instance.
% Be critical to what you read. Is paper quality assured, e.g., a peer-reviewed research paper? Are the assumptions appropriate for your task? Go into depth; a collection of "glossy" white-papers is insufficient

\cite{Author:year:XYZ} % just as an example to have a References section

\section{Methodology}
\label{sec:methodology}
% **Research** methodology
% Describe the overall process you intend to go through to achieve your results and assure their correctness and validity
% Present the method(s) and tool(s) that you plan to use and justify of their selection
% Discuss limitations/assumptions/hypotheses, if relevant, and possibly design choices that you have made (in which case you should justify them) or will have to make


% \section{Preliminary results}
% \label{sec:preliminary_results}
% Provide a short summary of what you have done, and present and explain the most relevant results

% \section{Ethics / privacy concerns}
% \label{sec:ethics}

\section{Project description and plan}
\label{sec:project_description_and_plan}
% Include a list of tasks (~5) (description, start-stop, planned effort in hours or days, concrete milestones (~1 per month), and dependencies, external and between tasks.
% Note that a plan is not cut in stone but should be a living document helping you to keep track of your effort and progress and enable controlled adjustments.

% \section{Concluding remarks}
% \label{sec:concluding_remarks}
% List for instance the main challenges that you foresee and/or summarize the decisions (choice of method/tool, design choices…) that you have made based on the preliminary study that you have conducted.  
  
\bibliographystyle{alpha}
\bibliography{main}

\end{document}
